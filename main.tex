\documentclass[10pt]{IEEEtran}
\usepackage[utf8]{inputenc}

\title{CS 6000 Journal 4}
\author{David Stout}
\date{September 2018}

\begin{document}

\maketitle

\section{Learning Process for a Survey rough draft}
Much of the process of this paper so far has been slightly confusing. It is something that 
is truly outside of my normal comfort zone for papers and has therefore become a bit hard at
times to find what I need to move forward.

In selecting a topic, I found that at first my topic was too broad and that I was 
overwhelmed by the amount of references I found. With a very broad topic I was having a very
hard time refining a cohesive and interesting story. So I tried to narrow my topic down, and
found that now my topic was too narrow to find enough sources and make a worthwhile story. 
After finally settling on a topic that I felt was "just right" in scope, I was able to start
really examining my sources.

At first, I found that I was spending far too much time trying to read the reference papers 
that I want to use. But after wasting time with the first few, I was able to get back into 
the groove of browsing and scanning that we learned in Journal 2. I would skim the abstract,
figures, conclusion and maybe the introduction to find the relevance to my topic and pick 
out something useful to my paper. After narrowing down my sources, I went back to the 
lecture and attempted to make sure that I was consciously trying to find the seven points 
that were needed for an example to be useful. This did take a bit longer than just simply 
scanning the papers, but I found that as I started to view more papers I was able to pick 
out the points much quicker.

Once I had enough relevant sources as solid examples I went about attempting to craft my 
story, and I found that I had terrible writers block. I would sit at the Overleaf screen and
write a sentence, to just turn around and delete it. I tried to concentrate on other things,
like learning how to ensure that the format was IEEE and that the font was in 10pt, in the 
attempt to try to get the creative juices flowing but nothing seemed to help. So I went back
to the lecture again and tried to reproduce some of the exercises that we tried in class. I 
closed my eyes and just typed whatever came to me. It seemed as though when I started that 
nothing was really helping, but the more I typed the more things started to click. I wrote a
few disconnected paragraphs, but it was a good start. I tried to create a storyboard to help
cement the ideas in a solid foundation, and help connect the dots.

From there I was able to begin piecing together a very rough outline of what I wanted to 
explore and how the sources that I had come up with will help me to show that there is a 
need for future research in the topic field as well as a handful of potential starting 
points for future research.

The hardest thing for me, so far, is excepting that this is just a rough draft and is far 
from done or close to perfect. It is merely a good place to begin and will need a lot of 
work and editing to create the paper that I feel it should be.

\end{document}
